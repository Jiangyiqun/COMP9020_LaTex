%Name: Template of COMP9020 Assignments
%Author:Jack
%Date: 14/08/2017
%Acknowledgement: This template is based on work of Brendan Trinh of UNSW MathSoc 2015
\documentclass[11pt, a4paper]{article}

\usepackage{amsmath} % Improves structure of typed out maths
\usepackage{mathtools} % Improves upon deficiencies of amsmath package
\usepackage{amssymb} % Adds some handy symbols to use.
\usepackage{amsthm} % Adds some neat formulas to use, e.g. \begin{proof} etc.

\usepackage[a4paper]{geometry} % Default page margins can be altered.
\usepackage{microtype} % Improves spacing between letters.
\usepackage{booktabs} % Improves tables. Can now create without vertical separators.
\usepackage{array} % Includes more options for arrays
\usepackage{paralist} % More flexible use of itemize, enumerate, etc.
\usepackage{graphicx} % Add images to your document
\usepackage{color} % Allows for the use of colours!
\usepackage{cleveref} % Better cross-referencing
\usepackage{hyperref} % For adding hyperlinks
\usepackage{fancyhdr} % Customise headers & footers in document
\usepackage{paralist}

\usepackage{url} % For adding url

\begin{document}
\title{COMP9020 - Assignment 3}
\author{Jack Jiang (z5129432)}
\date{ 17 October 2017 }
\maketitle
\graphicspath{{Graphics/}}

\section*{Question 1}
    On the left side:\\
    \\
    $ \binom nk = \frac{n!}{(n-k)!k!} = \frac {\frac {n!}{(n-k)}}{(n-k-1)!k!} $\\
    \\
    $ \binom {n}{k+1} = \frac {n!}{(n-k-1)!(k+1)!} = \frac {\frac {n!}{(k+1)}}{(n-k-1)!k!} $\\
    \\
    $ \binom nk + \binom {n}{k+1} = \frac {\frac {n!}{(n-k)} + \frac{n!}{(k+1)} }{(n-k-1)!k!}$\\
    \\
    On the right side:\\
    \\
    $ =  \frac {\frac {(n+1)n!}{(n-k)(k+1)} }{(n-k-1)!k!} = \frac { \frac {(n+1)!}{(n-k)(k+1)} }{(n-k-1)!(k)!} $\\
    \\
    $ \binom {n+1}{k+1} = \frac {(n+1)!}{(n-k)!(k+1)!}  =  \frac { \frac {(n+1)!}{(n-k)(k+1)} }{(n-k-1)!(k)!} $\\
    \\
    Therefore, left side is equal to right side
\section*{Question 2}
\begin{enumerate}[(a)]
    \item    
        let set X = \{$p$, $\neg p$, $q$\}\\
        let set Y = \{$\land$, $\lor$\}\\
        without using parenthesis, formulas can be constructed like this:\\
        X1 Y1 X2 Y2 X3\\
        the choices is:\\
        $3! \times 2! = 12$\\
        if we use 1 pair of parenthesis, we can put it there:\\
        (X1 Y1 X2) Y2 X3\\
        X1 Y1 (X2 Y2 X3)\\
        (X1 Y1 X2 Y2 X3)\\
        if we use 2 pairs of parenthesis, we can put them there:\\
        ((X1 Y1 X2) Y2 X3)\\
        (X1 Y1 (X2 Y2 X3))\\
        add up them together, there are $1+3+2=6$ methods to put parenthesis\\
        therefore, there are $12 \times 6 = 72$ different wff.
    \item
        there are 6 logical equivalence, as follows:\\
        $X \lor p$\\
        $X \lor q$\\
        $X \lor \neg p$\\
        $X \land p$\\
        $X \land q$\\
        $X \land \neg p$\\
        X is the combination of the other 3 symbols, based on Commutativity, X is unique.

    \end{enumerate}

\section*{Question 3}
\begin{enumerate}[(a)]
    \item 
\end{enumerate}

\section*{Question 4}
\begin{enumerate}[(a)]
    \item 
\end{enumerate}

\section*{Question 5}
\begin{enumerate}[(a)]
    \item 
\end{enumerate}

\end{document}