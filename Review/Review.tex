%Name: Template of COMP9020 Assignments
%Author:Jack
%Date: 14/08/2017
%Acknowledgement: This template is based on work of Brendan Trinh of UNSW MathSoc 2015
\documentclass[11pt, a4paper]{article}

\usepackage{amsmath} % Improves structure of typed out maths
\usepackage{mathtools} % Improves upon deficiencies of amsmath package
\usepackage{amssymb} % Adds some handy symbols to use.
\usepackage{amsthm} % Adds some neat formulas to use, e.g. \begin{proof} etc.

\usepackage[a4paper]{geometry} % Default page margins can be altered.
\usepackage{microtype} % Improves spacing between letters.
\usepackage{booktabs} % Improves tables. Clan now create without vertical separators.
\usepackage{array} % Includes more options for arrays
\usepackage{paralist} % More flexible use of itemize, enumerate, etc.
\usepackage{graphicx} % Add images to your document
\usepackage{color} % Allows for the use of colours!
\usepackage{cleveref} % Better cross-referencing
\usepackage{hyperref} % For adding hyperlinks
\usepackage{fancyhdr} % Customise headers & footers in document
\usepackage{paralist}

\usepackage{url} % For adding url

\begin{document}
\title{COMP9020 - Review}
\author{Jack Jiang (z5129432)}
\date{ 23 October 2017 }
\maketitle
\graphicspath{{/}}

\section*{Topic 1: Numbers, Sets, and Alphabets}
\begin{enumerate}
    \item Floor and ceiling
        \begin{itemize}
            \item $\lfloor \rfloor$: floor
            \item $\lceil \rceil$: ceiling
            \item $\lfloor -X \rfloor = - \lceil X \rceil$
            \item $\lfloor X + t \rfloor = \lfloor X \rfloor + t$ for t $\in$ Z
        \end{itemize}
    \item Divisibility, prime, gcd and lcm
        \begin{itemize}
            \item $m \mid n$: m divides n (m is less)
            \item $n \mid 0$ is true, and $0 \mid n$ is false, except n = 0
            \item prime: $n > 1$ and $1 \mid n$ and $n \mid n$ only
            \item relatively prime: $gcd(m, n) = 1$
            \item gcd: greatest common divisor
            \item lcm: least common multiple
            \item gcd(m, n) * lcm(m, n) = $|m| * |n|$
            \item Euclid's gcd algorithm: for m > n, gcd(m, n) = gcd(m-n, n)
        \end{itemize}
    \item Set notation and construction
        \begin{itemize}
            \item a set is a set of elements
            \item Notation 1: S = \{e1, e1, e1 ... \}
            \item Notation 2: S = \{e: description of e\}
            \item symmetric difference 1: $A \oplus B = (A \cup B) \setminus (A \cap B)$
            \item symmetric difference 2: $A \oplus B = (A \setminus B) \cup (B \setminus A)$
            \item Subset: $\subseteq$, Proper subset: $\Subset$
            \item Power set: Pow(X) = $\{A: A \subseteq X\}$
            \item Cardinality: $|X|$
            \item Always: $|Pow(X)| = 2^{|X|}$
            \item Set of Numbers: P $\subset$ N $\subset$ Z $\subset$ Q $\subset$ R
        \end{itemize}
    \item Laws of Sets Operations
        \begin{itemize}
            \item Commutativity
            \item Associativity
            \item Distribution
            \item Idempotence
            \item Identity
            \item Double Complementation
            \item De morgan Laws: $(A \cup B))^C = A^C \cap B^C$, $(A \cap B))^C = A^C \cup B^C$
        \end{itemize}
    \item Cartesian product
        \begin{itemize}
            \item (a, b): ordered pair
            \item $A \times B = \{(a,b) | a \in A, b \in B\}$
        \end{itemize}
    \item Formal language
        \begin{itemize}
            \item $\Sigma$: alphabet -- a finite, none empty set
            \item $\lambda$: a empty word
            \item $\Sigma^k$: set of all words of length k
            \item $\Sigma^*$: set of all words
            \item $\Sigma^+$: set of all none empty words
        \end{itemize}
    \end{enumerate}

\section*{Topic 2: Functions Matrix and Relations}
    \begin{enumerate}
        \item Function Definition
            \begin{itemize}
                \item notation 1: $f: S \rightarrow T$
                \item notation 2: $f: x \mapsto y$
                \item notation 3: f(x) = y
                \item every input has an one and only one output
                \item Image: Im(f) = $\{f(x), x \in Dom(f)\}$
                \item $Im(f) \subset Codom(f)$
                \item Composition: $g \circ f = g(f(x))$ where Im(f) $\subset$ Dom(g)
                \item Identity: $f \circ Id = Id \circ f = f$
            \end{itemize}
        \item Function inverse
            \begin{itemize}
                \item surjective(onto): every output has a related input\\
                    $$Im(f) = Codom(f)$$
                \item injective(one-to-one): every input has an unique output\\
                    $$x \ne y \implies f(x) \ne f(y)$$
                    $$f(x) = f(y) \implies x = y$$
                \item bijective\\
                    $$surjective and injective$$
                \item inverse\\
                    $$f^{-1}: y \rightarrow x$$
                \item $f: D \rightarrow C, S_D \subseteq D, S_C \subseteq C$, then:\\
                    $f(S_D) \subseteq C$ is the image, and $f^{\Leftarrow}(S_C) \subseteq D$ is the inverse image\\
                    if $f^{-1}(S_C) = f^{\Leftarrow}(S_C)$
            \end{itemize}
        \item Matrix
            \begin{itemize}
                \item $M_{mn}$ m is row and n is column
                \[
                    \begin{bmatrix}
                        m_{11}  &  m_{12}  &  \dots  &  m_{1n}\\
                        m_{21}  &  m_{22}  &  \dots  &  m_{2n}\\
                        \dots\\
                        m_{m1}  &  m_{m2}  &  \dots  &  m_{mn}   
                    \end{bmatrix}
                \]
                \item Transpose $M^{T}$\\
                a matrix is called symmetric if $M^T = M$
                \item Sum
                \item product (first row second column)
                \[
                    \begin{bmatrix}
                        a_{11}  &  a_{12}   &  a_{13}\\
                        a_{21}  &  a_{22}   &  a_{23}\\
                    \end{bmatrix}
                    \times
                    \begin{bmatrix}
                        b_{11}  &  b_{12}\\
                        b_{21}  &  b_{22}\\
                        b_{31}  &  b_{32}\\
                    \end{bmatrix}            
                    =
                    \begin{bmatrix}
                        a_{11}*b_{11}+a_{12}*b_{21}+a_{13}*b_{31}  &  a_{11}*b_{12}+a_{12}*b_{22}+a_{13}*b_{32}\\
                        a_{21}*b_{21}+a_{22}*b_{21}+a_{13}*b_{31}  &  a_{21}*b_{12}+a_{22}*b_{22}+a_{23}*b_{32}\\
                    \end{bmatrix}                       
                \]
            \end{itemize}
        \item
            \begin{itemize}
                \item c
            \end{itemize}
        \end{enumerate}

\section*{Topic 3: Graph theory}
    \begin{enumerate}
        \item
        \item
    \end{enumerate}

\section*{Topic 4: Logic}
    \begin{enumerate}
        \item
        \item
    \end{enumerate}

\section*{Topic 5: Induction}
    \begin{enumerate}
        \item
        \item
    \end{enumerate}

\section*{Topic 6: Recursion}
    \begin{enumerate}
        \item
        \item
    \end{enumerate}

\section*{Topic 7: Running time of programs}
    \begin{enumerate}
        \item
        \item
    \end{enumerate}

\section*{Topic 8: Counting}
    \begin{enumerate}
        \item
        \item
    \end{enumerate}

\section*{Topic 9: Probability and Expectation}
    \begin{enumerate}
        \item
        \item
    \end{enumerate}

\end{document}