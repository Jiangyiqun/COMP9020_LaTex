%Name: Template of COMP9020 Assignments
%Author:Jack
%Date: 14/08/2017
%Acknowledgement: This template is based on work of Brendan Trinh of UNSW MathSoc 2015
\documentclass[11pt, a4paper]{article}

\usepackage{amsmath} % Improves structure of typed out maths
\usepackage{mathtools} % Improves upon deficiencies of amsmath package
\usepackage{amssymb} % Adds some handy symbols to use.
\usepackage{amsthm} % Adds some neat formulas to use, e.g. \begin{proof} etc.

\usepackage[a4paper]{geometry} % Default page margins can be altered.
\usepackage{microtype} % Improves spacing between letters.
\usepackage{booktabs} % Improves tables. Can now create without vertical separators.
\usepackage{array} % Includes more options for arrays
\usepackage{paralist} % More flexible use of itemize, enumerate, etc.
\usepackage{graphicx} % Add images to your document
\usepackage{color} % Allows for the use of colours!
\usepackage{cleveref} % Better cross-referencing
\usepackage{hyperref} % For adding hyperlinks
\usepackage{fancyhdr} % Customise headers & footers in document
\usepackage{paralist}

\usepackage{url} % For adding url

\begin{document}
\title{COMP9020 - Assignment 1}
\author{Jack(z5129432)}
\date{ 14ed August 2017 }
\maketitle

% \section*{Set notation}
% \begin{enumerate}[(a)]
% 	$$ \O $$
% 	$$ \in $$
% 	$$ \notin $$
% 	$$ \subset $$
% 	$$ \cup $$
% 	$$ \cap $$
% 	$$ \setminus $$
% 	$$ \oplus $$
% 	$$  $$
% \end{enumerate}

\section*{Question 1}
\begin{enumerate}[(a)]
	\item
	we have $$gcd(288, 120) = gcd(120, 48) = (48, 24) = (24, 24)$$
	so $$gcd(288, 120) = 24$$
	\item
	we know that $$lcm(-91, 52) = lcm(91, 52)$$
	first calculate $$gcd(91, 52) = gcd(39, 52) = gcd(13, 39) = gcd(13, 13) = 13$$
	according to \cite{lcm} $$gcd(91, 52) \times lcm(91, 52) = 91 \times 52 = 4732$$ 
	we can know that $$lcm(-91, 52) = \frac{4732}{gcd(91, 52)} = \frac{4732}{13} = 364$$
	\item
	$$ \because n + 1 > n \, for \, n \in N $$
	according to Euclid's algorithm \cite{euclid}
	$$gcd(n +1, n) = gcd (n + 1 - n, n) = gcd(n, 1)$$
	$\therefore$ n and n+1 are relative prime, for $ n \in N$
\end{enumerate}

\section*{Question 2}
\begin{enumerate}[(a)]
	\item 
	$$ \because \, Pow(\emptyset) = \emptyset $$
	$$ \therefore \, Pow(Pow(\emptyset)) = \emptyset$$
	$$ Card( \emptyset) = 0 $$

	\item

\end{enumerate}

\section*{Question 3}
\begin{enumerate}[(a)]
	\item solution to q1.a)
	\item solution to q1.b)
\end{enumerate}

\section*{Question 4}
\begin{enumerate}[(a)]
	\item solution to q1.a)
	\item solution to q1.b)
\end{enumerate}

\section*{Question 5}
\begin{enumerate}[(a)]
	\item solution to q1.a)
	\item solution to q1.b)
\end{enumerate}

\begin{thebibliography}{99}
\bibitem{lcm}
\url{https://en.wikipedia.org/wiki/Least_common_multiple#Fundamental_theorem_of_arithmetic} 
\bibitem{euclid}
\url{https://en.wikipedia.org/wiki/Greatest_common_divisor#Using_Euclid.27s_algorithm}
\end{thebibliography}

\end{document}